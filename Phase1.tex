\documentclass[12pt,a4paper,oneside]{article}
\setlength{\emergencystretch}{5em}

% ============= Packages =============
\usepackage[a4paper,
bindingoffset=0in,
left=1in,
right=1in,
top=1in,
bottom=1in,
footskip=.5in]{geometry}
\usepackage[table, svgnames, dvipsnames, xcdraw]{xcolor}
\usepackage{calc}
\usepackage{makecell, cellspace, caption}
\usepackage{eso-pic}
\usepackage{enumitem}
\usepackage{hyperref}
\usepackage[]{graphicx}
\usepackage{pifont}
\usepackage{float}
\usepackage{fontawesome5}
\usepackage{longtable}
\usepackage{tabularx}
\usepackage{amssymb}
\usepackage{pdfpages}
\usepackage{xepersian}

% ============= Border =============
\newlength{\PageFrameTopMargin}
\newlength{\PageFrameBottomMargin}
\newlength{\PageFrameLeftMargin}
\newlength{\PageFrameRightMargin}

\setlength{\PageFrameTopMargin}{1cm}
\setlength{\PageFrameBottomMargin}{1cm}
\setlength{\PageFrameLeftMargin}{1cm}
\setlength{\PageFrameRightMargin}{1cm}

\makeatletter

\newlength{\Page@FrameHeight}
\newlength{\Page@FrameWidth}

\AddToShipoutPicture{
	\thinlines
	\setlength{\Page@FrameHeight}{\paperheight-\PageFrameTopMargin-\PageFrameBottomMargin}
	\setlength{\Page@FrameWidth}{\paperwidth-\PageFrameLeftMargin-\PageFrameRightMargin}
	\put(\strip@pt\PageFrameLeftMargin,\strip@pt\PageFrameTopMargin){
		\framebox(\strip@pt\Page@FrameWidth, \strip@pt\Page@FrameHeight){}}}

\makeatother


\settextfont{Vazirmatn-Regular}
\setlatintextfont{Arial}
\linespread{1.5}
\defpersianfont\secondFont{Vazirmatn-Regular}
\definecolor{lightsexypurple}{HTML}{DDA0DD}
\newcolumntype{C}[1]{>{\centering\arraybackslash}m{#1}}

% ============= Main Document Section =============
\begin{document}
	% ========== Title Page ==========
	\begin{titlepage}
		\centering
		\includegraphics[width=0.35\textwidth]{resources/logo}
		\bigbreak
		{\large دانشگاه اصفهان}
		
		{\large دانشکده مهندسی کامپیوتر}
		
		\vspace{\baselineskip}
		{\Huge\textbf{ دادیار}}
		
		{\normalsize سامانه هوشمند ادله دیجیتال}
		
		\vspace{2\baselineskip}
		\textbf{پدیدآورندگان به ترتیب الفبا:}
		
		مهرزاد انصاری پور
		
		ابوالفضل دشتی اردکانی
		
		علی صالحی

		امرحسین مرادی	

		ستایش ورعی یگانه

		طه یوسفی گورتی
		
		\vspace{\baselineskip}
		{\textbf{گروه} \textbf{{12}}}
		
		\vspace{2\baselineskip}
		{\textbf{استاد راهنما:}}
		دکتر محمدرضا شعرباف
		
		{\textbf{دستیار آموزشی:}}
		سرکار خانم مهدیه ترابی 	
		
		\vspace{3\baselineskip}
		
		زمستان
		{{1403}}
		
	\end{titlepage}
	\cleardoublepage

    	% ========== TOC ==========
        \pagestyle{empty}
        \tableofcontents
        \clearpage
        \pagestyle{plain}

        % Start of the doc

	\newpage
	\section{سند تبیین نیازمندی‌ها}
	\vspace{-2em} 
	\par\noindent\rule{\textwidth}{0.72pt}
	
	\subsection{مقدمه}	
		با گسترش فناوری و سهولت دسترسی به سیستم‌های دیجیتال، نیاز به سامانه‌های امن و کارا برای کمک به تحقق
	حقوق افراد و سازمان‌ها احساس می‌شود. پیشرفت هوش مصنوعی و سامانه‌های دیجیتال علاوه بر امکانات مفیدی
	که در اختیار انسان قرار می‌دهد می‌تواند به ابزار سودجویان برای نیل به اهداف نامشروع واقع شود. با چنین شرایطی،
	مطلوب است سامانه‌هایی طراحی و ساخته شود که با توسعه در لبه علم‌وفناوری، بتوانند احقاق حقوق افراد را تسهیل
	 کنند.	سند حاضر به تبیین نیازمندی‌های چنین سامانه‌ای 
		در ایران می‌پردازد این سند بر اساس استاندارد 
		\lr{\hyperref[ref:ieee]{\textbf{IEEE}} \hyperref[ref:std]{\textbf{Std}} 830-1998}
		تدوین شده است.
		\subsubsection{هدف}
		هدف از پیاده‌سازی سامانه دادیار، استفاده از دانش و تکنولوژی روز برای مدیریت و پایش ادله دیجیتال و ارجاع آن‌ها به مراجع قضایی است. اهداف اصلی این سامانه عبارت‌اند از:
		
		\begin{itemize}
			\item تسهیل و تسریع فرآیند جمع‌آوری، ثبت و مدیریت ادله دیجیتال برای شاکیان، متشاکیان و مسئولین پرونده
			\item بهبود دقت و صحت فرآیندهای قضایی با استفاده از الگوریتم‌های پیشرفته هوش مصنوعی
			\item کاهش هزینه‌ها و زمان مورد نیاز برای بررسی و رسیدگی به پرونده‌های قضایی
			\item کاهش احتمال خطاهای انسانی در تحلیل داده‌های پرونده‌ها
			\item ایجاد یک پایگاه داده امن و جامع برای مدیریت مستندات دیجیتال
		\end{itemize}

		به طور خلاصه، \hyperref[ref:srs]{\textbf{SRS}} (سند تبیین نیازمندی‌های نرم‌افزار) دیدی جامع از محصول نهایی را به تصویر می‌کشد
		علاوه بر این، \hyperref[ref:srs]{\textbf{SRS}} در ابتدای پروژه به‌عنوان مبنایی برای پیش‌بینی زمان‌بندی و برآورد هزینه‌های پروژه مورداستفاده قرار می‌گیرد.
			
		\subsubsection{قلمرو}

			سامانه مدیریت ادله دیجیتال \textbf{"دادیار"} با بهره‌گیری از الگوریتم‌های بهینه و \hyperref[ref:ai]{\textbf{هوش مصنوعی}}، نقش مهمی در صحت‌سنجی مستندات شکایات ایفا می‌کند. این سامانه با هوشمند\-سازی فرآیندهای سنتی، تشریفات اداری را مدیریت و از بروز تخلفات جلوگیری می‌کند.  
			\textbf{دادیار} با استفاده از الگوریتم‌های پیشرفته، قابلیت تشخیص جعل و دستکاری در ادله را بهبود بخشیده و دقت بررسی‌های حقوقی را افزایش می‌دهد. همچنین، با بهره‌گیری از \hyperref[ref:gui]{\textbf{رابط کاربری}} آسان و کاربرپسند، امکان مدیریت سریع و دقیق‌ اطلاعات را برای کاربران فراهم می‌کند. با این حال، این سامانه جایگزین قاضی نیست و تصمیم‌گیری نهایی همچنان بر عهده مقام قضایی خواهد بود.  
			علاوه بر تسهیل امور اداری، \textbf{دادیار} به عنوان بستری کارآمد برای بهینه‌سازی فرآیندهای حقوقی، افزایش دقت و کاهش زمان رسیدگی به پرونده‌ها، تجربه‌ای سریع‌ و دقیق‌ را برای مراجع قضایی فراهم می‌آورد.

			\subsubsection{تعاریف، سرنام‌ها و کوته‌نوشت‌ها}

			\begin{itemize}
				\item 
				\textbf{IEEE\rl{:}\label{ref:ieee}}( کوتاه شده عبارت
				\lr{Institute of Electrical and Electronics Engineer}
				)موسسه مهندسان برق و الکترونیک
		
				\item 
				\textbf{SRS\rl{:}\label{ref:srs}}
				(کوتاه شده عبارت 
				\lr{Software Requirement Specification}) 
				سند تبیین نیازمندی های نرم افزار است.
				
				\item
				\textbf{Std\rl{:}\label{ref:std}}
				(کوتاه شده Standard است)
				استاندارد به مجموعه‌ای از قوانین، مقررات و الزامات گفته می‌شود که برای یک محصول، فرایند یا سیستم خاص تعریف می‌شود.

				\item
					\textbf{الگوریتم:\label{ref:algorithm}}
					الگوریتم مجموعه‌ای از دستورالعمل‌های گام‌به‌گام برای حل یک مسئله یا انجام یک وظیفه خاص است. به‌عبارت‌دیگر، الگوریتم به رایانه می‌گوید برای انجام یک کار مشخص چه مراحلی را باید انجام دهد.

					\item 
					\textbf{هوش مصنوعی:\label{ref:ai}}
					هوش مصنوعی به شبیه‌سازی فرآیندهای هوشمند انسانی توسط سیستم‌های کامپیوتری می‌شود که شامل یادگیری، استدلال و خوداصلاحی است.

					\item 
					\textbf{:API\label{ref:api}}
					(کوتاه شده
					\lr{Application Programming Interface}
					)رابط  کاربری بین برنامه های مجزا است.
					 \item 
					 \textbf{\lr{Native App}:\label{ref:nativeapp}} 
					 برنامه‌هایی هستند که برای یک سیستم‌عامل خاص مانند اندروید، iOS یا ویندوز طراحی و توسعه یافته‌اند.

					 \item 
					 \textbf{روتر:\label{ref:router}}
					  روتر دستگاهی است که ترافیک شبکه را مسیریابی می‌کند، اتصال به اینترنت را به اشتراک می‌گذارد و امنیت شبکه را افزایش می‌دهد.
					  
					  \item 
					  \textbf{سیستم‌عامل:\label{ref:os}}
					  سیستم‌عامل (\lr{Operating System}) نرم‌افزاری است که منابع سخت افزاری را مدیریت و برنامه های کاربردی را اجرا می کند.

					  \item 
					  \textbf{\lr{Web Server}:\label{ref:webserver}} 
					  نرم‌افزاری است که در یک سیستم رایانه‌ای اجرا می‌شود و وظیفه آن پاسخگویی به درخواست‌های HTTP از سوی کاربران وب است. به عبارتی دیگر، وب سرور واسطی بین مرورگر وب کاربر و تارنما است.
					  
					    \item 
					  \textbf{\lr{Android, IOS}: \label{ref:androidios}}
					  دو سیستم‌عامل محبوب برای گوشی‌های هوشمند هستند که هر کدام مزایا و معایب خاص خود را دارند.

					  \item 
					  \textbf{مرورگر وب:\label{ref:browser}}
					   نوعی نرم‌افزار کاربردی است مثل 
					   \lr{Google Chrome} ،\lr{Microsoft Edge} و \lr{FireFox}
						که برای دریافت، نمایش، مرور و ارسال اطلاعات، جستجوی تارنماها در وب جهانی یا یک تارنمای محلی مورداستفاده قرار می‌گیرد.
						
						\item 
					  \textbf{سرور ابری:\label{ref:cloudserver}}
					   یک نوع سرور است که در رایانش ابری ایجاد شده و بر روی بستر اینترنت برای بسیاری از کاربران ارائه می‌شود. 
					   
					   \item 
					    \textbf{طراحی واکنش‌گرا:\footnote{\lr{Responsive Design}}\label{ref:responsivedesign}} رویکردی در طراحی تارنما است که باعث می‌شود تارنماها بر روی همه اندازه‌های صفحه‌نمایش، از گوشی‌های هوشمند کوچک گرفته تا نمایشگرهای بزرگ رایانه، به‌خوبی نمایش داده شوند.

						\item
					  \textbf{ کارت شبکه:\label{ref:networkcard}}
						به نام‌های آداپتور شبکه و مبدل شبکه نیز شناخته می‌شود، یک قطعه سخت‌افزاری است که به رایانه شما اجازه می‌دهد تا به شبکه‌های رایانه‌ای متصل شود.

						\item
					   \textbf{HTML\rl{:}\label{ref:html}}
					   یک‌زبان نشانه‌گذاری است که کوتاه شده عبارت
					    \lr{Hyper Text Markup Language}
					     است.
					     
					   \item
					   \textbf{\lr{CSS}:\label{ref:css}}
					    یک‌زبان نشانه‌گذاری است که کوتاه شده عبارت
					     \lr{Cascading Style Sheets}
					      است.
					      
					   \item
					   \textbf{JavaScript\rl{:}\label{ref:js}}
به‌اختصار JS یک‌زبان برنامه‌نویسی است که برای توسعه نرم‌افزارهای مرتبط با وب استفاده می‌شود.
					    
					    \item 
					   \textbf{\lr{MySQL}:\label{ref:mysql}}
					   یک 
					   پایگاه‌داده رابطه‌ای
					   \footnote{\lr{Relational Database}\label{ref:rd}}
					    متن‌باز و محبوب است که به دلیل سادگی، کارایی و انعطاف‌پذیری بالا، به طور گسترده در وبگاه‌ها و برنامه‌های وب استفاده می‌شود.
					    
					    \item 
					   \textbf{MongoDB\rl{:}\label{ref:mongo}}
					   یک پایگاه‌داده NoSQL است که از مدل سند برای ذخیره‌سازی داده‌ها استفاده می‌کند. این نوع پایگاه‌داده برای ذخیره‌سازی داده‌های غیرساختاریافته مانند اسناد JSON و XML ایده آل است.
					   
					   \item 
					    \textbf{Oracle\rl{:}\label{ref:oracle}}
					    اوراکل یک سیستم مدیریت پایگاه‌داده رابطه‌ای یا RDBMS است که توسط شرکت اوراکل توسعه‌یافته است. این سیستم یکی از محبوب‌ترین پایگاه‌های داده در جهان است و توسط سازمان‌های بزرگ و کوچک در سراسر دنیا استفاده می‌شود.

					   \item
					   \textbf{واسط گرافیک کاربری:\label{ref:gui}}
					    واسط گرافیک کاربری یا GUI مخفف
					     \lr{Graphical User Interface}
					    ، نوعی رابط کاربری است که به‌جای استفاده از متن، از عناصر گرافیکی مانند آیکون‌ها، دکمه‌ها، منو و پنجره‌ها برای تعامل کاربر با رایانه استفاده می‌کند.

						 \item 
						\textbf{HTTPS\label{ref:https}}
						 کوتاه شده عبارت 
						 \lr{Hyper Text Transfer Protocol Secure}
						  است که یک پروتکل ارتباطی برای انتقال امن اطلاعات در شبکه‌های رایانه‌ای است که به‌صورت خاص در اینترنت استفاده می‌شود.
						  
						\item 
						\textbf{پروتکل:\label{ref:protocol}}
						 پروتکل به معنی مجموعه از قوانین و رویه‌ها برای برقراری ارتباط است.
						 
						\item 
						\textbf{TLS\label{ref:tls}}
						 مخفف 
						 \lr{Transport Layer Security}
						  (به معنی امنیت لایه انتقال) است. این پروتکل امنیتی برای محافظت از ارتباطات در برابر شنود، جعل هویت و دست‌کاری داده‌ها در اینترنت استفاده می‌شود.
	  
						  \item
						  \textbf{روان‌شناسی رنگ‌ها:\label{ref:colorpsychology}}
						   روان‌شناسی رنگ‌ها مطالعه تأثیر رنگ بر درک و رفتار انسان است. رنگ‌ها می‌توانند احساسات و واکنش‌های فیزیکی مختلفی را در افراد برانگیزند. به‌عنوان‌مثال، رنگ قرمز می‌تواند با هیجان و خشم مرتبط باشد، درحالی‌که رنگ آبی می‌تواند با آرامش و صلح مرتبط باشد. روان‌شناسی رنگ‌ها را می‌توان در زمینه‌های مختلفی از جمله بازاریابی، طراحی و هنر استفاده کرد.
						  
						  \item
						  \textbf{Microservices\rl{:}\label{ref:microservice}}
						یک الگوی معماری نرم‌افزاری است که در آن یک برنامه کاربردی پیچیده به بخش‌های کوچک و مستقلی به نام میکروسرویس تقسیم می‌شود. هر میکروسرویس مسئولیت مشخصی را برعهده دارد و به‌صورت مستقل از سایر بخش‌ها توسعه، استقرار، مقیاس‌بندی و نگه‌داری می‌شود.
						  
						  \item 
						  \textbf{هش‌کردن:\label{ref:hash}}
						   هش‌کردن (Hashing) فرایندی است که طی آن یک ورودی با طول دلخواه به یک خروجی با طول ثابت تبدیل می‌شود. این خروجی که به آن هش (Hash) یا کد هش \lr{(Hash Code)} گفته می‌شود، یک‌رشتهٔ منحصربه‌فرد از کاراکترها است. فرایند هش‌کردن شبیه به فشرده‌سازی اطلاعات عمل می‌کند، اما با این تفاوت که بازیابی اطلاعات اولیه از روی هش به‌سادگی ممکن نیست. این فرایند برای ذخیره رمزهای عبور در پایگاه‌داده‌ها استفاده می‌شود.
				
						   \item
						   \textbf{رمزنگاری:\label{ref:encryption}}
							دانشی است که به بررسی و شناخت اصول و روش‌های تبدیل اطلاعات به رمز، به‌منظور حفظ امنیت و محرمانگی آن می‌پردازد.
							 				 
							\item 
							\textbf{پایگاه‌داده:\label{ref:database}}
							به مجموعه‌ای از داده‌ها با ساختار منظم و سامان‌مند گفته می‌شود.

							\item 
							\textbf{RAM\rl{:}\label{ref:ram}}
							کوتاه شده
							 \lr{Random Access Memory}
							 به معنی حافظه دسترسی تصادفی است. رم نوعی حافظه رایانه‌ای است که برای ذخیره موقت داده‌ها و کدها استفاده می‌شود. این نوع حافظه به طور مستقیم توسط پردازنده (CPU) قابل‌دسترسی است و سرعت بالایی دارد.
							 
							 \item
							\textbf{CPU\rl{:}\label{ref:cpu}}
							کوتاه شده عبارت
							 (\lr{Central Processing Unit})
							  است که به معنی واحد پردازش مرکزی است. CPU را می‌توان مغز متفکر رایانه در نظر گرفت. این قطعه الکترونیکی، مسئول اجرای دستورالعمل‌های نرم‌افزارها و پردازش داده‌ها است.	
							  
							  \item 
								\textbf{شاکی:\label{ref:complainant}} 
								خواهان، فردی که اقامه دعوا می‌کند.
								
								\item 
								\textbf{متشاکی:\label{ref:defendant}} 
								مشتکی‌عنه، خوانده، فردی که از وی شکایت شده و به دادگاه خوانده می‌شود.
								
								\item 
								\textbf{مسئول پرونده:\label{ref:caseofficer}} 
								در اینجا منظور قضات و تمام کسانی است که کار داوری و دادرسی را انجام می‌دهند.
								
								\item 
								\textbf{مدیر سامانه:\label{ref:admin}} 
								فرد یا افرادی که پشتیبانی سامانه را انجام می‌دهند و نقشی در فرآیند دادرسی ندارند.
														

			\end{itemize}
		
			\subsubsection{مراجع}

			\begin{itemize}
				\item 
				\lr{Software engineering: a practitioner's approach, Pressman, Roger S. Palgrave macmillan, 2005}
				
				\item 
				کونگ، دیوید سی: مهندسی نرم‌افزار شئ‌گرا (یک متدولوژی چابک یکنواخت) جلد اوّل. ترجمه: دکتر بهمن زمانی و دکتر افسانه فاطمی، ۱۳۹۴
			
			\end{itemize}

			\subsubsection{طرح کلی}
			در این سند، ابتدا یک نگاه اجمالی به روند دادرسی بر اساس ادله الکترونیکی و دیجیتالی و اهداف سامانه \textbf{دادیار} خواهیم داشت. سپس به معرفی کلی اجزای سیستم، از جمله انواع کاربران، نحوه ارتباط آن‌ها با سامانه، واسط‌های مختلف نرم‌افزاری و سخت‌افزاری و همچنین زیرساخت‌های موردنیاز می‌پردازیم. در ادامه، قوانین و محدودیت‌های حقوقی مرتبط با این سامانه بررسی خواهند شد. در نهایت، به تشریح جزئیات و تبیین نیازمندی‌های گوناگون سیستم، از جمله نیازمندی‌های کارکردی، غیرکارکردی، قیود طراحی، الزامات امنیتی و... خواهیم پرداخت. همچنین، ویژگی‌ها و صفات مطلوب مورد انتظار از سامانه \textbf{دادیار} نیز در این بخش موردبحث قرار می‌گیرند.

	\subsection{شرح کلی}

	\subsubsection{چشم‌انداز محصول}

سامانه "دادیار" باهدف هوشمندسازی سیستم مدیریت و تحلیل ادله دیجیتال طراحی شده است. این سامانه با ارائه خدمات نوین و استفاده از فناوری‌های پیشرفته، به دنبال ایجاد تجربه‌ای کارآمد و دقیق برای کاربران خود است.
یکی از اهداف اصلی این سامانه، تسهیل و تسریع در جمع‌آوری، تحلیل و مدیریت ادله دیجیتال برای سازمان‌ها و نهادهای قانونی است. از امکانات این سامانه می‌توان به:

\begin{itemize}
\item پردازش و تحلیل خودکار مستندات دیجیتال و تشخیص تخلفات
\item امکان تشخیص جعل و دستکاری مستندات با استفاده از هوش مصنوعی
\item ارائه گزارش‌های تحلیلی و داشبورد مدیریتی برای مقامات قضایی
\item یکپارچگی با پایگاه‌های داده قضایی برای بهبود دسترسی به اطلاعات پرونده‌ها
\end{itemize}
اشاره کرد.در ادامه، واسط‌هایی که سامانه "دادیار" به آنها نیاز دارد را بیان می‌کنیم.
			
	\begin{enumerate}
		\item 
		\textbf{واسط‌های سیستم\footnote{\lr{System Interfaces}}}
		\\
		در این بخش، ارتباط سامانه دادیار با سیستم‌های خارجی و روش انتقال اطلاعات بین این سیستم‌ها را بررسی می‌کنیم.
		\begin{itemize}
			\item ارتباط با سامانه‌های قضایی موجود مانند سامانه ثبت احوال و سامانه دادگاه‌ها جهت احراز هویت و دریافت اطلاعات پرونده‌ها
		\end{itemize}

		\item 
	\textbf{واسط‌های کاربر\footnote{\lr{User Interfaces}}}
	 در سامانه دادیار، کاربران باید بتوانند با اتصال به شبکه اینترنت از هر دو طریق تارنما و 
	 \lr{\hyperref[ref:nativeapp]{\textbf{Native App}}}
	  باتوجه‌به نقش و سطح دسترسی از سامانه استفاده کنند. همچنین رابطه کاربری دادیار باید به‌گونه‌ای طراحی شود تا اعضای تازه‌وارد نیز بتوانند به‌راحتی درخواست خود را ثبت کنند و بدون نیاز به آموزش‌های تخصصی از این سامانه استفاده کنند.


	\item 
	\textbf{واسط‌های سخت‌افزاری\footnote{\lr{Hardware Interfaces}}}
	\\
	دادیار به واسط‌های سخت‌افزاری زیر نیاز دارد:
	
	\begin{itemize}
		\item
		تجهیزات برای دسترسی به اینترنت مانند 
		\hyperref[ref:router]{\textbf{روتر}}
		، 
		\hyperref[ref:networkcard]{\textbf{کارت شبکه}} 
		و تلفن همراه با قابلیت اتصال به اینترنت
		
		\item
		گوشی هوشمند با 
		\hyperref[ref:os]{\textbf{سیستم‌عامل}} 
		\lr{\hyperref[ref:androidios]{\textbf{IOS}}}
		 یا 
		\lr{\hyperref[ref:androidios]{\textbf{Android}}} 
		برای اجرای نرم‌افزار 
		دادیار یا هر دستگاه دیگری که قابلیت اجرای 
		\hyperref[ref:browser]{\textbf{مرورگرهای‌وب}}
		 را 
		برای دسترسی به تارنما دادیار داشته باشد.
		
		\item
		\hyperref[ref:cloudserver]{\textbf{سرور ابری}}
		 قدرتمند با 
		\lr{\hyperref[ref:cpu]{\textbf{CPU}}}
		،
		\lr{\hyperref[ref:ram]{\textbf{RAM}}}
		 و حافظه کافی برای 
		 \hyperref[ref:database]{\textbf{پایگاه‌داده}}
		 ،
		 \lr{\hyperref[ref:webserver]{\textbf{Web Server}}}
		  و 
		 \lr{\hyperref[ref:api]{API}}
		  نرم‌افزار 
		 \lr{IOS}
		  و 
		  \lr{Android}

	\end{itemize}


	\item 
	\textbf{واسط‌های نرم‌افزاری\footnote{\lr{Software Interfaces}}}
		\begin{itemize}
			\item
							ازآنجاکه این سیستم بر بستر اینترنت است، کاربر به 
			\hyperref[ref:browser]{\textbf{مرورگرهای وب}}
			از جمله
			\lr{Google Chrome}، \lr{Firefox}، \lr{Microsoft Edge}
			و یا هر مرورگر دیگری که از 
			\lr{\hyperref[ref:html]{\textbf{HTML}} ,\hyperref[ref:css]{\textbf{CSS}}} و \lr{\hyperref[ref:js]{\textbf{JavaScript}}}
			پشتیبانی کند؛ نیاز دارد.
			
			\item
			 ازآنجاکه این سامانه به‌صورت روزانه
			با حجم بالایی از داده‌ها سروکار دارد؛ به یک سیستم 
			مدیریت پایگاه‌داده مانند
			\lr{\hyperref[ref:mongo]{\textbf{MongoDB}}, \hyperref[ref:mysql]{\textbf{MySQL}}, \hyperref[ref:oracle]{\textbf{Oracle}}}
			و … نیاز دارد.
			
			\item
			   نرم‌افزار و تارنما این سامانه 
			\hyperref[ref:responsivedesign]{\textbf{طراحی واکنش‌گرا}}
			دارند و امکان تغییر اندازه عناصر واسط گرافیک کاربری بر روی همه دستگاه‌هایی که صفحه‌نمایش‌هایی با اندازه‌های مختلف دارند؛ نمایش داده می‌شود.
		\end{itemize}

		\item 
		\textbf{واسط‌های ارتباطی\footnote{\lr{Communication Interfaces}}}
		\\
		دادیار در همه بخش‌ها از جمله 
		\lr{\hyperref[ref:webserver]{\textbf{Web Server}}}
		 و
		  \hyperref[ref:api]{\textbf{API}}
		  نرم‌افزار از پروتکل 
		  \hyperref[ref:https]{\textbf{HTTPS}}
		   بر بستر 
		   \hyperref[ref:tls]{\textbf{TLS}}
			استفاده می‌کند تا همه اطلاعات کاربران در حین انتقال 
			\hyperref[ref:encryption]{\textbf{رمزگذاری}}
			 شده باشد. این سامانه از پیامک و پست الکترونیک برای ارسال اعلان‌ها استفاده می‌کند. در این سامانه هر کاربر بسته به سطح دسترسی رابط کاربری مخصوص خود را دارد.

		\item 
		\textbf{واسط‌های حافظه\footnote{\lr{Memory Interfaces}}}

		\begin{itemize}
				\item سامانه باید از زیرساخت‌های ذخیره‌سازی ابری برای مدیریت حجم بالای داده‌ها استفاده کند.
			     \item سامانه باید امکان پشتیبان‌گیری دوره‌ای از داده‌ها را داشته باشد.
			     \item سامانه باید امکان دسترسی سریع و امن به داده‌ها را فراهم کند.
			     \item سامانه باید از الگوریتم‌های بهینه برای مدیریت حافظه و جلوگیری از ازدحام داده‌ها استفاده کند.
		\end{itemize}

		\item 
		\textbf{واسط‌های عملیات\footnote{\lr{Operations Interfaces}}}
		 واسط‌های عملیات شامل فرآیندها و وظایفی است که برای اجرای صحیح سامانه دادیار لازم است. این واسط‌ها به شرح زیر می‌باشند:

		\begin{itemize}
				\item سامانه باید امکان مدیریت و نگهداری داده‌ها را داشته باشد.
				\item سامانه باید امکان بازیابی اطلاعات در صورت بروز خطا یا نقص فنی را فراهم کند.
				\item سامانه باید امکان نظارت و کنترل بر عملیات‌های روزانه را به مدیران سامانه بدهد.
				\item سامانه باید امکان اجرای وظایف خودکار برای به‌روزرسانی داده‌ها و نرم‌افزارها را داشته باشد.
				\item ارسال هشدارها و اعلان‌های خودکار به کاربران و مقامات قضایی
				\item پردازش تصویری و متنی برای تشخیص جعل و تغییرات غیرمجاز
		\end{itemize}

		\item 
		\textbf{نیازمندی‌های سازگاری با محل نصب}
		\\
		همان‌طور که در واسط‌های نرم‌افزاری نیز اشاره شد؛ تارنما سامانه دادیار روی تمام دستگاه‌های دارای 
		\hyperref[ref:browser]{\textbf{مرورگر}}
		، قابل‌اجرا است و نیاز به نصب ندارد.

	\end{enumerate}

	\subsubsection{کارکرد محصول}

	سیستم در کل شامل ویژگی‌های زیر است:
	
	\begin{itemize}
		\item 
		سامانه امکان بارگذاری مدارک و مستندات را برای شاکیان و متشاکیان فراهم می‌کند.
	
		\item 
		سامانه امکان پیگیری روند پرونده را برای شاکیان و متشاکیان فراهم می‌کند.
	
		\item 
		سامانه با بهره‌گیری از هوش مصنوعی، جعل ادله را شناسایی و گزارش می‌کند.
	
		\item 
		سامانه با استفاده از هوش مصنوعی مانع ثبت‌نام و ورود ربات‌ها می‌شود.
	
		\item 
		سامانه با مدیریت هوشمند مستندات و ادله دیجیتال، ازدحام و شلوغی مراکز قضایی را کاهش می‌دهد.
	
		\item 
		سامانه باعث صرفه‌جویی در وقت و هزینه کاربران می‌شود.
	
		\item 
		سامانه در مصرف کاغذ و منابع طبیعی صرفه‌جویی می‌کند.
	
		\item 
		سامانه امکان ثبت مستندات را نسبت به روش‌های مرسوم بهینه‌سازی کرده و روند دادرسی را کاهش می‌دهد.
	
		\item 
		سامانه این امکان را می‌دهد تا کاربران از طریق پیامک و ایمیل بتوانند از وضعیت لحظه‌ای پرونده و سایر اطلاعات مربوطه مطلع شوند.
	\end{itemize}
	
			\subsubsection{مشخصات کاربران}

			دادیار دارای چهار نوع کاربر به شرح زیر است: 			
			\begin{itemize}
				\item 
				\textnormal{\large \hyperref[ref:plaintiff]{شاکی (خواهان)}}
				
				که به‌طورکلی شامل این موارد است:
				\begin{itemize}
					\item 
					موظف به آشنایی با فرآیند احراز هویت
					\item 
					امکان بارگذاری مدارک و مستندات با داشتن شماره پرونده قضایی
					\item 
					دریافت اطلاع‌رسانی درباره زمان و مکان دادگاه
					\item 
					...
				\end{itemize}
			
				\item 
				\textnormal{\large \hyperref[ref:defendant]{متشاکی (خوانده)}}
				
				که به‌طورکلی شامل این موارد است:
				\begin{itemize}
					\item 
					موظف به آشنایی با فرآیند احراز هویت
					\item 
					امکان ایجاد حساب کاربری در صورت عدم وجود با استفاده از سامانه دولت من
					\item 
					امکان بارگذاری مدارک و مستندات با داشتن شماره پرونده قضایی
					\item 
					دریافت اطلاع‌رسانی درباره زمان و مکان دادگاه
					\item 
					...
				\end{itemize}
			
				\item 
				\textnormal{\large \hyperref[ref:case_officer]{مسئول پرونده}}
				
				که به‌طورکلی شامل این موارد است:
				\begin{itemize}
					\item 
					آشنایی با قوانین و مقررات قضایی
					\item 
					بررسی و قضاوت مستندات پرونده
					\item 
					تعیین حکم نهایی دادگاه
					\item 
					ثبت رأی نهایی در سامانه و اطلاع‌رسانی به کاربران مرتبط
					\item 
					...
				\end{itemize}
			
				\item 
				\textnormal{\large \hyperref[ref:admin]{مدیر سامانه}}
				
				که به‌طورکلی شامل این موارد است:
				\begin{itemize}
					\item 
					داشتن دانش فنی مدیریت و توسعه سیستم‌های دیجیتال
					\item 
					امکان ارتباط و پشتیبانی کاربران
					\item 
					...
				\end{itemize}
			\end{itemize}
			
			\subsubsection{قیود و محدودیت‌ها}
			\begin{enumerate}
				\renewcommand{\labelenumi}{\textbf{.C\arabic{enumi}}}
				\item 
				سامانه باید در تمام ساعات شبانه‌روز در دسترس باشد.
			
				\item 
				سامانه باید از الگوریتم‌ها و پروتکل‌های امنیتی مناسب برای حفاظت از اطلاعات کاربران استفاده کند.
			
				\item 
				سامانه نیاز به تیم پشتیبانی (فنی و ارتباطی) در صورت بروز مشکل دارد.
			
				\item 
				سامانه باید زمان‌بندی دقیق و معینی برای تحویل مستندات به مراجع قضایی داشته باشد.
			
				\item 
				رابط کاربری سامانه باید برای عموم مردم قابل درک باشد.
			
				\item 
				سامانه نیازمند استفاده از الگوریتم‌های مناسب 
				\hyperref[ref:ai]{\textbf{هوش مصنوعی}}
				جهت صحت‌سنجی مدارک و مستندات است.
			
				\item 
				سامانه به دلیل نیاز به اطلاع‌رسانی، باید از زیرساخت‌های مناسب در جهت اطلاع‌رسانی و ارسال پیام استفاده کند.
			
				\item 
				سامانه نیاز به درگاه پرداخت امن و مطمئن جهت پرداخت هزینه دارد.
			\end{enumerate}
			
			\subsubsection{مفروضات و وابستگی‌ها}

			\textbf{مفروضات:}
			\begin{itemize}
				\item 
				کاربران (شاکی، متشاکی و مسئول پرونده) باید توانایی خواندن، نوشتن و کار با اپلیکیشن سامانه یا وب را داشته باشند.
			
				\item 
				کاربران باید دسترسی به تلفن همراه، رایانه شخصی و اینترنت پرسرعت داشته باشند.
			
				\item 
				شرکت‌ها برای فعالیت در سامانه باید دارای مجوزهای قانونی لازم از مراجع ذی‌صلاح باشند.
			
				\item 
				کاربران باید اطلاعات هویتی مانند نام، نام خانوادگی، کد ملی، پست الکترونیک و … را در سامانه ثبت کنند.
			
				\item 
				فرض می‌کنیم که شاکی و متشاکی می‌توانند مستندات خود را تا زمان اعلام حکم از سوی دادگاه در سامانه ثبت کنند.
			
				\item 
				فرض می‌کنیم که شهروندان دارای تابعیت ایران می‌توانند از سامانه استفاده نمایند.
			\end{itemize}
			
			\textbf{وابستگی‌ها:}
			\begin{itemize}
				\item 
				سامانه برای پردازش حجم بالای دستورات نیازمند پردازنده‌های قدرتمند است.
			
				\item 
				سامانه برای ذخیره اطلاعات و مستندات نیازمند استفاده از تجهیزات ذخیره‌سازی با حجم مناسب است.
			
				\item 
				سامانه نیازمند دسترسی به اطلاعات سایر سامانه‌های مرتبط مثل دولت من است.
			
				\item 
				سامانه باید بتواند از پرداخت آنلاین پشتیبانی کند. برای این منظور باید از API سامانه شاپرک
				استفاده شود.
			
				\item 
				سامانه باید بتواند پیامک‌های اطلاع‌رسانی به کاربران ارسال کند. در این راستا باید از  سرویس‌های پیامکی
				استفاده کند.
			
				\item 
				سامانه باید از پروتکل‌های امنیتی مناسب استفاده کند.
			
				\item 
				برای تضمین پایداری، سامانه باید استفاده بهینه‌ای از زیرساخت‌های شبکه داشته باشد.
			
				\item 
				در صورت ارائه خدمات تخصصی مانند پیگیری و صحت‌سنجی مستندات سایبری و امنیتی، به دانش و مجوزهای لازم وابسته است.
			\end{itemize}
				
	\subsection{نیازمندی‌های خاص}
	سیستم دارای نیازمندی‌های متعددی است که بر اساس خواسته مشتری در ادامه به طور کامل شرح داده شده‌اند.

	\subsubsection{نیازمندی‌های کارکردی}
در این بخش قابلیت‌های سیستم باتوجه‌به نوع و سطح دسترسی کاربران طبقه‌بندی شده‌اند.

\begin{itemize}
    \item
    \textbf{نیازمندی‌های مدیر سامانه}
    \begin{enumerate}
        \renewcommand{\labelenumi}{\textbf{.R\arabic{enumi}}}

        \item 
        سیستم باید امکان مدیریت کاربران را برای مدیریت سامانه فراهم کند.
        \begin{enumerate}
            \renewcommand{\labelenumii}{\textbf{.R\arabic{enumi}.\arabic{enumii}}}
            \item 
            سیستم باید امکان فعال یا غیر فعال کردن حساب‌های کاربری را به مدیر سامانه بدهد.
            \item 
            سیستم باید امکان تعیین سطوح دسترسی برای کاربران را داشته باشد.
        \end{enumerate}

        \item 
        سیستم باید امکان مدیریت ادله دیجیتال را داشته باشد.
        \begin{enumerate}
            \renewcommand{\labelenumii}{\textbf{.R\arabic{enumi}.\arabic{enumii}}}
            \item 
            سیستم باید امکان اتمام رسیدگی توسط مدیران را داشته باشد.
            \item 
            سیستم باید گزارش رسیدگی به مدیران بدهد.
        \end{enumerate}

        \item 
        سیستم باید امکان پشتیبانی برای مدیران سامانه داشته باشد.
        \begin{enumerate}
            \renewcommand{\labelenumii}{\textbf{.R\arabic{enumi}.\arabic{enumii}}}
            \item 
            سیستم باید امکان ثبت و پیگیری درخواست پشتیبانی برای مدیران سامانه را فراهم کند.
            \item 
            سیستم باید امکان گزارش وضعیت درخواست به کاربران را داشته باشد.
        \end{enumerate}

        \item 
        سیستم باید امکان رفع شکایات فنی توسط مدیران سامانه را داشته باشد.
        \begin{enumerate}
            \renewcommand{\labelenumii}{\textbf{.R\arabic{enumi}.\arabic{enumii}}}
            \item 
            سیستم باید امکان بروز رسانی نرم افزاری را داشته باشد.
        \end{enumerate}

        \item 
        سیستم باید امکان تحلیل و گزارش مناسب به مدیران سامانه را داشته باشد.
        \begin{enumerate}
            \renewcommand{\labelenumii}{\textbf{.R\arabic{enumi}.\arabic{enumii}}}
            \item 
            سیستم باید امکان ایجاد داشبوردهای تحلیلی برای مدیران سامانه داشته باشد.
            \begin{enumerate}
                \renewcommand{\labelenumiii}{\textbf{.R\arabic{enumi}.\arabic{enumii}.\arabic{enumiii}}}
                \item 
                سیستم باید امکان تحلیل روند شکایت و تراکنش برای مدیران سامانه فراهم کند.
            \end{enumerate}
            \item 
            سیستم باید امکان گزارش از منابع سرور برای مدیران سامانه داشته باشد.
        \end{enumerate}

        \item 
        سیستم باید امکان مدیریت ارتباط مدیران و پشتیبانان سامانه با کاربر را داشته باشد.
        \begin{enumerate}
            \renewcommand{\labelenumii}{\textbf{.R\arabic{enumi}.\arabic{enumii}}}
            \item 
            سیستم باید قابلیت پیگیری سابقه تعاملات کاربران توسط مدیر سامانه را داشته باشد.
            \item 
            سیستم باید امکان ارسال اطلاعیه و پیام مدیران به کاربران را داشته باشد.
        \end{enumerate}

        \item 
        سیستم باید امکان مدیریت انتقادات و پیشنهادها توسط مدیران سیستم را داشته باشد.
        \begin{enumerate}
            \renewcommand{\labelenumii}{\textbf{.R\arabic{enumi}.\arabic{enumii}}}
            \item 
            سیستم باید امکان پاسخگویی مدیران به انتقادات را داشته باشد.
        \end{enumerate}
    \end{enumerate}

    \item
    \textbf{نیازمندی‌های شاکی}
    \begin{enumerate}
        \renewcommand{\labelenumi}{\textbf{.R\arabic{enumi}}}
        \setcounter{enumi}{7}

        \item 
        سیستم باید امکان ثبت‌نام را برای کاربر فراهم سازد.
        \begin{enumerate}
            \renewcommand{\labelenumii}{\textbf{.R\arabic{enumi}.\arabic{enumii}}}
            \item 
            سیستم باید امکان بارگذاری مدارک لازم برای ثبت‌نام اولیه شامل نام و نام خانوادگی، پست الکترونیک، شماره‌تلفن همراه، نشانی محل سکونت و رمز ملی به کاربر حقیقی بدهد.
        \end{enumerate}

        \item 
        سیستم باید به کاربرانى که ثبت نام خود را تکمیل کرده‌اند امکان ورود به حساب کاربری را دهد.
        \begin{enumerate}
            \renewcommand{\labelenumii}{\textbf{.R\arabic{enumi}.\arabic{enumii}}}
            \item 
            سیستم باید امکان ورود به سامانه با کد یک بار مصرف را به کاربر بدهد.
            \item 
            سیستم باید امکان ورود به سامانه با رمز عبور را به کاربر بدهد.
        \end{enumerate}

        \item 
        سیستم باید امکان ویرایش اطلاعات شخصى را به کاربر بدهد.
        \begin{enumerate}
            \renewcommand{\labelenumii}{\textbf{.R\arabic{enumi}.\arabic{enumii}}}
            \item 
            سیستم باید امکان ثبت و ویرایش اطلاعات شخصى از جمله نام و نام خانوادگی، شماره تلفن همراه، ایمیل، جنسیت، رمز عبور و کد ملى را به مشتری بدهد.
        \end{enumerate}

        \item 
        سیستم باید امکان بارگذاری مستندات تصویر، فیلم، متن و صوت را به کاربر بدهد.
        \begin{enumerate}
            \renewcommand{\labelenumii}{\textbf{.R\arabic{enumi}.\arabic{enumii}}}
            \item 
            سیستم باید امکان بارگذاری مستندات در فرمت‌های مختلف را به کاربر بدهد.
        \end{enumerate}

        \item 
        سیستم باید امکان صحت سنجی مستندات را داشته باشد.
        \begin{enumerate}
            \renewcommand{\labelenumii}{\textbf{.R\arabic{enumi}.\arabic{enumii}}}
            \item 
            سیستم باید تصاویر را با استفاده از تکنیک‌های پردازش تصویر بررسی کند و تصاویر افراد را بر اساس مدل‌های شناسایی اشیا استخراج کند و جهت استفاده در سامانه قرار دهد.
            \item 
            سیستم باید صدای افراد را بر اساس تکنیک‌های پردازش صوت بررسی کند و نتیجه شناسایی افراد را در سامانه قرار دهد.
        \end{enumerate}

        \item 
        سیستم باید امکان به‌روزرسانی اطلاعات پرونده را قبل از تأیید مسئول پرونده به کاربر بدهد.
        \begin{enumerate}
            \renewcommand{\labelenumii}{\textbf{.R\arabic{enumi}.\arabic{enumii}}}
            \item 
            سیستم باید امکان افزودن اطلاعات را به کاربر بدهد.
        \end{enumerate}

        \item 
        سیستم باید امکان اطلاع‌رسانی لحظه‌به‌لحظه پرونده را به کاربر بدهد.

        \item 
        سیستم باید امکان جستجوی پرونده‌ها را به کاربر بدهد.
        \begin{enumerate}
            \renewcommand{\labelenumii}{\textbf{.R\arabic{enumi}.\arabic{enumii}}}
            \item 
            سیستم باید امکان جستجو پرونده‌ها را بر اساس شماره پرونده، وضعیت پرونده و زمان شکایت به کاربر بدهد.
        \end{enumerate}

        \item 
        سیستم باید امکان دیدن نتیجه رأی نهایی را به کاربر بدهد.

        \item 
        سیستم باید امکان ارتباط مسئول پرونده و کاربر را داشته باشد.
        \begin{enumerate}
            \renewcommand{\labelenumii}{\textbf{.R\arabic{enumi}.\arabic{enumii}}}
            \item 
            سیستم باید امکان تبادل پیام بین کاربر و مسئول پرونده را داشته باشد.
            \item 
            سیستم باید امکان اعتراض به رأی مسئول پرونده را به کاربر بدهد.
            \item 
            سیستم باید امکان تقاضا برای تجدیدنظر نسبت به صحت سنجی مدارک توسط مسئول مربوطه را به کاربر بدهد.
        \end{enumerate}

        \item 
        سیستم باید امکان مشاهده پرونده‌های مختومه را به کاربر بدهد.

        \item 
        سیستم باید امکان بازیابی اطلاعات پرونده‌های کاربر را به شاکی بدهد.
        \begin{enumerate}
            \renewcommand{\labelenumii}{\textbf{.R\arabic{enumi}.\arabic{enumii}}}
            \item 
            سیستم باید امکان بازیابی اطلاعات ازدست‌رفته و خراب را بدهد.
            \item 
            سیستم باید امکان ذخیره سوابق پرونده‌های قبل را برای کاربر فراهم سازد.
        \end{enumerate}

        \item 
        سیستم باید امکان ارائه راهنمایی‌های لازم برای تکمیل شکایت را داشته باشد.

        \item 
        سیستم باید امکان درخواست حذف شکایت را به کاربر بدهد.

		\item 
        سیستم باید امکان اعلان هوشمند برای اطلاع‌رسانی به شاکی را داشته باشد.
        \begin{enumerate}
            \renewcommand{\labelenumii}{\textbf{.R\arabic{enumi}.\arabic{enumii}}}
            \item 
            سیستم باید تغییرات پرونده را به شاکی اطلاع‌رسانی کند.

			\item 
			سیستم باید مستندات شاکی را به‌صورت رمز شده دربیاورد.

        \end{enumerate}


    \end{enumerate}

    \item
    \textbf{نیازمندی‌های متشاکی}
    \begin{enumerate}
        \renewcommand{\labelenumi}{\textbf{.R\arabic{enumi}}}
        \setcounter{enumi}{22}

		
        \item 
        سیستم باید امکان ثبت‌نام را برای کاربر فراهم سازد.
        \begin{enumerate}
            \renewcommand{\labelenumii}{\textbf{.R\arabic{enumi}.\arabic{enumii}}}
            \item 
            سیستم باید امکان بارگذاری مدارک لازم برای ثبت‌نام اولیه شامل نام و نام خانوادگی، پست الکترونیک، شماره‌تلفن همراه، نشانی محل سکونت و رمز ملی به کاربر حقیقی بدهد.
        \end{enumerate}

        \item 
        سیستم باید به کاربرانى که ثبت نام خود را تکمیل کرده‌اند امکان ورود به حساب کاربری را دهد.
        \begin{enumerate}
            \renewcommand{\labelenumii}{\textbf{.R\arabic{enumi}.\arabic{enumii}}}
            \item 
            سیستم باید امکان ورود به سامانه با کد یک بار مصرف را به کاربر بدهد.
            \item 
            سیستم باید امکان ورود به سامانه با رمز عبور را به کاربر بدهد.
        \end{enumerate}

        \item 
        سیستم باید امکان ویرایش اطلاعات شخصى را به کاربر بدهد.
        \begin{enumerate}
            \renewcommand{\labelenumii}{\textbf{.R\arabic{enumi}.\arabic{enumii}}}
            \item 
            سیستم باید امکان ثبت و ویرایش اطلاعات شخصى از جمله نام و نام خانوادگی، شماره تلفن همراه، ایمیل، جنسیت، رمز عبور و کد ملى را به مشتری بدهد.
        \end{enumerate}

        \item 
        سیستم باید امکان بارگذاری مستندات تصویر، فیلم، متن و صوت را به کاربر بدهد.
        \begin{enumerate}
            \renewcommand{\labelenumii}{\textbf{.R\arabic{enumi}.\arabic{enumii}}}
            \item 
            سیستم باید امکان بارگذاری مستندات در فرمت‌های مختلف را به کاربر بدهد.
        \end{enumerate}

        \item 
        سیستم باید امکان صحت سنجی مستندات را داشته باشد.
        \begin{enumerate}
            \renewcommand{\labelenumii}{\textbf{.R\arabic{enumi}.\arabic{enumii}}}
            \item 
            سیستم باید تصاویر را با استفاده از تکنیک‌های پردازش تصویر بررسی کند و تصاویر افراد را بر اساس مدل‌های شناسایی اشیا استخراج کند و جهت استفاده در سامانه قرار دهد.
            \item 
            سیستم باید صدای افراد را بر اساس تکنیک‌های پردازش صوت بررسی کند و نتیجه شناسایی افراد را در سامانه قرار دهد.
        \end{enumerate}

        \item 
        سیستم باید امکان به‌روزرسانی اطلاعات پرونده را قبل از تأیید مسئول پرونده به کاربر بدهد.
        \begin{enumerate}
            \renewcommand{\labelenumii}{\textbf{.R\arabic{enumi}.\arabic{enumii}}}
            \item 
            سیستم باید امکان افزودن اطلاعات را به کاربر بدهد.
        \end{enumerate}

        \item 
        سیستم باید امکان اطلاع‌رسانی لحظه‌به‌لحظه پرونده را به کاربر بدهد.

        \item 
        سیستم باید امکان جستجوی پرونده‌ها را به کاربر بدهد.
        \begin{enumerate}
            \renewcommand{\labelenumii}{\textbf{.R\arabic{enumi}.\arabic{enumii}}}
            \item 
            سیستم باید امکان جستجو پرونده‌ها را بر اساس شماره پرونده، وضعیت پرونده و زمان شکایت به کاربر بدهد.
        \end{enumerate}

        \item 
        سیستم باید امکان دیدن نتیجه رأی نهایی را به کاربر بدهد.

        \item 
        سیستم باید امکان ارتباط مسئول پرونده و کاربر را داشته باشد.
        \begin{enumerate}
            \renewcommand{\labelenumii}{\textbf{.R\arabic{enumi}.\arabic{enumii}}}
            \item 
            سیستم باید امکان تبادل پیام بین کاربر و مسئول پرونده را داشته باشد.
            \item 
            سیستم باید امکان اعتراض به رأی مسئول پرونده را به کاربر بدهد.
            \item 
            سیستم باید امکان تقاضا برای تجدیدنظر نسبت به صحت سنجی مدارک توسط مسئول مربوطه را به کاربر بدهد.
        \end{enumerate}

        \item 
        سیستم باید امکان مشاهده پرونده‌های مختومه را به کاربر بدهد.

        \item 
        سیستم باید امکان بازیابی اطلاعات پرونده‌های کاربر را به شاکی بدهد.
        \begin{enumerate}
            \renewcommand{\labelenumii}{\textbf{.R\arabic{enumi}.\arabic{enumii}}}
            \item 
            سیستم باید امکان بازیابی اطلاعات ازدست‌رفته و خراب را بدهد.
            \item 
            سیستم باید امکان ذخیره سوابق پرونده‌های قبل را برای کاربر فراهم سازد.
        \end{enumerate}	

        \item 
        سامانه باید امکان ارسال دفاعیه توسط کاربر را فراهم کند.

        \item 
        سامانه باید قابلیت ارسال درخواست‌های مشاوره قانونی را برای کاربر فراهم کند.


		\item 
        سیستم باید امکان اعلان هوشمند برای اطلاع‌رسانی به متشاکی را داشته باشد.
        \begin{enumerate}
            \renewcommand{\labelenumii}{\textbf{.R\arabic{enumi}.\arabic{enumii}}}
            \item 
            سیستم باید تغییرات پرونده را به متشاکی اطلاع‌رسانی کند.
        \end{enumerate}


		\item 
        سیستم باید مستندات متشاکی را به‌صورت رمز شده دربیاورد.
    \end{enumerate}

    \item
    \textbf{نیازمندی‌های مسئول رسیدگی به پرونده}
    \begin{enumerate}
        \renewcommand{\labelenumi}{\textbf{.R\arabic{enumi}}}
        \setcounter{enumi}{38}

        \item 
        سیستم باید امکان اتصال پرونده‌های مرتبط را به یکدیگر به مسئول پرونده بدهد.

        \item 
        مسئول پرونده باید امکان ورود به‌حساب کاربری خود را داشته باشد.
        \begin{enumerate}
            \renewcommand{\labelenumii}{\textbf{.R\arabic{enumi}.\arabic{enumii}}}
            \item 
            مسئول پرونده باید بعد از تأیید دومرحله‌ای به سامانه وارد شود.
        \end{enumerate}

        \item 
        مسئول پرونده باید امکان مشاهده پرونده‌های خود را داشته باشد.
        \begin{enumerate}
            \renewcommand{\labelenumii}{\textbf{.R\arabic{enumi}.\arabic{enumii}}}
            \item 
            مسئول پرونده باید امکان مشاهده اطلاعات شاکی، متشاکی و مستندات پرونده را داشته باشد.
        \end{enumerate}

        \item 
        سیستم باید امکان تحلیل مستندات و ارائه گزارش متنی را به مسئول پرونده بدهد.

        \item 
        سیستم باید امکان برسی صحت و اعتبارسنجی پرونده را برای مسئول مربوطه فراهم سازد.

        \item 
        سیستم باید امکان ثبت فعالیت و گزارش‌های پرونده را به مسئول پرونده بدهد.
        \begin{enumerate}
            \renewcommand{\labelenumii}{\textbf{.R\arabic{enumi}.\arabic{enumii}}}
            \item 
            سیستم باید امکان اطلاع‌رسانی مراحل پرونده به شاکی، متشاکی و ما فوق را به مسئول پرونده بدهد.
            \item 
            سیستم باید امکان ذخیره گزارش‌های پرونده را داشته باشد.
        \end{enumerate}

        \item 
        سیستم باید قابلیت ارتباط مسئول پرونده با شاکی، مشتکی‌عنه، مافوق و مدیر سامانه را داشته باشد.
        \begin{enumerate}
            \renewcommand{\labelenumii}{\textbf{.R\arabic{enumi}.\arabic{enumii}}}
            \item 
            سیستم باید امکان مکاتبه و تعامل با شاکی، متشاکی و مدیر سامانه را برای مسئول پرونده فراهم سازد.
        \end{enumerate}

        \item 
        سیستم باید امکان تعلیق پرونده را در صورت کمبود مستندات به مسئول پرونده بدهد.
        \begin{enumerate}
            \renewcommand{\labelenumii}{\textbf{.R\arabic{enumi}.\arabic{enumii}}}
            \item 
            سیستم باید امکان درخواست مستندات بیشتر از شاکی و متشاکی را به مسئول پرونده بدهد.
        \end{enumerate}

        \item 
        سیستم باید امکان ارجاع پرونده به مسئولان مربوطه را به مسئول پرونده بدهد.
        \begin{enumerate}
            \renewcommand{\labelenumii}{\textbf{.R\arabic{enumi}.\arabic{enumii}}}
            \item 
            سیستم باید امکان ثبت دلایل ارجاع پرونده را به مسئول پرونده بدهد.
        \end{enumerate}

        \item 
        سیستم باید امکان مستندسازی انجام شده روی پرونده را فراهم سازد.
        \begin{enumerate}
            \renewcommand{\labelenumii}{\textbf{.R\arabic{enumi}.\arabic{enumii}}}
            \item 
            سیستم باید به‌صورت خودکار در هر مرحله از بررسی پرونده وقایع را ثبت کند.
        \end{enumerate}

        \item 
        سیستم باید امکان تغییر وضعیت پرونده را به مسئول پرونده بدهد.
        \begin{enumerate}
            \renewcommand{\labelenumii}{\textbf{.R\arabic{enumi}.\arabic{enumii}}}
            \item 
            سیستم باید امکان تغییر وضعیت پرونده به در حال بررسی، توقف بررسی و پایان بررسی را به مسئول پرونده بدهد.
        \end{enumerate}

		\item 
        سیستم باید امکان ارتباط با سایر سیستم‌های قضایی را داشته باشد.
        \begin{enumerate}
            \renewcommand{\labelenumii}{\textbf{.R\arabic{enumi}.\arabic{enumii}}}
            \item 
            سیستم باید امکان ارسال مستقیم مستندات را به مراجع قانونی مرتبط را داشته باشد.
		\end{enumerate}	

		\item 
        سیستم باید امکان ذخیره‌سازی تغییرات مسئول پرونده را داشته باشد.
        \begin{enumerate}
            \renewcommand{\labelenumii}{\textbf{.R\arabic{enumi}.\arabic{enumii}}}
            \item 
            سیستم باید نسخه‌های پیشین مستندات را ذخیره کند.
        \end{enumerate}

        \item 
        سیستم باید امکان ارائه گزارش‌های آماری از پرونده‌ها را داشته باشد.

    \end{enumerate}
	
\end{itemize}

			\subsubsection{نیازمندی‌های غیرکارکردی}
			نیازمندی‌های غیرکارکردی، ویژگی‌هایی از یک سیستم هستند که به عملکرد اصلی آن مرتبط نیستند، اما بر روی تجربه کاربری و احساس کلی کاربر از محصول تأثیر به سزایی می‌گذارند. این نیازها شامل امنیت، عملکرد، نگهداری، سازگاری، استفاده‌پذیری و مقیاس‌پذیری می‌شوند. این نیازمندی‌ها، اساسی برای معماری سیستم هستند و در زمان طراحی باید مدنظر قرار گیرند تا سامانه به‌درستی عمل کند و تجربه کاربری مناسبی را برای کاربران فراهم کند. اگر به نیازهای غیرکارکردی توجه نشود؛ احتمال ازدست‌دادن کاربران و جذب‌شدن آنها توسط سیستم‌های مشابه وجود دارد.
			\begin{itemize}
				\item \textbf{امنیت\footnote{Security}}
				\begin{itemize}
					\item \textbf{رمزنگاری داده‌ها:}
					استفاده از 
					\hyperref[ref:encryption]{\textbf{رمزنگاری}}
					 برای حفاظت از اطلاعات کاربران در زمان انتقال و ذخیره‌سازی اطلاعات.
					
					\item \textbf{احراز هویت دومرحله‌ای:}
					فراهم‌کردن امکان احراز هویت دومرحله‌ای برای ورود به سیستم، مانند ارسال کد تأیید به شماره‌تلفن همراه یا ایمیل ایمیل ملی در دولت من.
					
					\item \textbf{احراز هویت:}
					اطلاعات کاربران را از پایگاه‌داده‌های خود و یا دیگر پایگاه‌داده‌های در دسترس سیستم به دست آورد.
					
				\end{itemize}
				
				\item \textbf{عملکرد\footnote{Performance}}
				\begin{itemize}
					\item \textbf{زمان پاسخگویی:}
					عملکرد مناسب در زمان پاسخگویی به درخواست‌های کاربران و کاهش زمان بازکردن صفحات و همچنین امکان پاسخ‌گویی به کاربر در ۲۴ ساعت شبانه‌روز.
					
					\item \textbf{ظرفیت پردازش:}	
					تهیه مدل های آماری برای پیشبینی ترافیک سامانه و استفاده از منابع مناسب جهت جلوگیری از  افت سرعت سامانه.
				\end{itemize}
				
				\item \textbf{نگهداری\footnote{Maintainability}}
				\begin{itemize}
					\item \textbf{کد قابل توسعه:}
					استفاده از کد توسعه‌پذیر و قابل‌اصلاح برای افزودن و به‌روزرسانی ویژگی‌های جدید.
					
					\item \textbf{مستندسازی:}		
					مستندسازی کامل تمامی قسمت‌های سیستم برای آسان‌تر کردن فرایند نگهداری و توسعه.
					
					\item \textbf{مدیریت خطا:}
					\item درصورت بروز مشکل در سامانه ( سخت افزاری و نرم افزاری) باید فورا به اطلاع تیم پشتیبانی رسیده تا بتوانند اقدامات مناسب انجام دهند.
					\item اگر سیستم به هر دلیلی از دسترس کاربران خارج شد،  اطلاع رسانی کاربران باید به درستی انجام بشود.
					\item گزارش فرایند های انجام شده در سامانه باید ذخیره شوند.
				\end{itemize}
				
				\item \textbf{سازگاری \footnote{Interoperability}}
				\begin{itemize}
					\item \textbf{تطابق با استانداردها:}
					سازگاری با استانداردها و ارائه و پشتیبانی از روش‌ها و 
					\hyperref[ref:protocol]{\textbf{پروتکل}}های استاندارد برای ارتباطات بین سیستم و سرویس‌های خارجی به‌منظور ایجاد اتصالات سازگار و بدون مشکل.		
				\end{itemize}
				
				\item \textbf{استفاده‌پذیری \footnote{Usability}}
				\begin{itemize}
					\item \textbf{طراحی رابط کاربری:}		
					طراحی 
					\hyperref[ref:gui]{\textbf{رابط کاربری}}
					 برای استفاده آسان و قابل‌درک برای تمامی دسته‌های کاربران.
					
					
				\end{itemize}
				
				\item \textbf{مقیاس‌پذیری  \footnote{Scalability}}
				\begin{itemize}
					\item \textbf{قابلیت تغییر اندازه:}
					 قابلیت افزایش یا کاهش اندازه و نیازمندی‌ها از لحاظ نرم‌افزاری و سخت‌افزاری.
					 
					\item \textbf{مقیاس‌پذیری افقی و عمودی:}
					قابلیت افزایش تعداد کاربران و شدآمد
					\footnote{شدآمد، مصوب فرهنگستان برای ترافیک است.}
					 (مقیاس‌پذیری افقی) و افزایش نرخ‌پردازش (مقیاس‌پذیری عمودی).		
				\end{itemize}
				
			\end{itemize}

			\subsubsection{نیازمندی‌های واسط خارجی}
			نیازمندی‌های واسط خارجی مشخص می‌کنند چگونه 
			سیستم با عوامل خارجی ارتباط برقرار کند، 
			از جمله رابط‌های نرم‌افزاری، سخت‌افزاری و کاربری. 
			هدف این نیازمندی‌ها، اطمینان حاصل‌کردن از ارتباط 
			صحیح سیستم با اجزای خارجی است که می‌تواند تأثیر 
			مستقیمی بر عملکرد و کارایی سیستم داشته باشد. 
			رابط‌های نرم‌افزاری مانند 
			\hyperref[ref:api]{\textbf{API}} 
			\footnote{\lr{Application Programming Interface}}
			و 
			\hyperref[ref:webserver]{\textbf{وب‌سرویس‌ها}}
			، رابط‌های 
			سخت‌افزاری شامل درگاه‌ها
			\footnote{درگاه، مصوب فرهنگستان برای \lr{port} است.}
			 و دستگاه‌های سخت‌افزاری دیگر و
			 رابط‌های کاربری از قبیل 
			 \hyperref[ref:gui]{\textbf{واسط‌های گرافیکی}}
			  و دکمه‌ها هستند. توضیحات مربوط به نیازمندی‌های واسط خارجی از جمله رابط‌های سیستم 
			 در بخش چشم‌انداز محصول به طور کامل شرح داده شده است.

			 \subsubsection{قیود طراحی}
			 محدودیت‌های طراحی، مجموعه‌ای از شرایط هستند که بر فرایند طراحی تأثیر می‌گذارند و اغلب توسط مشتریان، نهادهای توسعه‌ای یا قوانین بین‌المللی اعمال می‌شوند.
			 \\
			 برخی از این قیود عبارت‌اند از:
			 \begin{itemize}
				 \item 
				 \textbf{سازگاری با دستگاه‌های مختلف:}
				 سیستم باید بر روی انواع دستگاه‌ها از جمله تلفن‌های همراه، تبلت‌ها و رایانه‌ها به‌خوبی کار کند. این امر از طریق 
				 \hyperref[ref:responsivedesign]{\textbf{طراحی واکنش‌گرا}}
				  و آزمون‌های متعدد دستگاهی تضمین می‌شود.
				 
				 \item 
				 \textbf{امنیت داده‌ها:}
				 محافظت از اطلاعات شخصی کاربران و جزئیات سفارش‌ها ضروری است. باید از 
				 \hyperref[ref:encryption]{\textbf{رمزنگاری}}
				  استاندارد و تأیید هویت چندعاملی استفاده شود تا از داده‌ها در برابر دسترسی‌های غیرمجاز محافظت شود.
				 
				 \item 
				 \textbf{رابط کاربری ساده و قابل‌دسترس:}
				 رابط کاربری باید طوری طراحی شود که استفاده از آن برای کاربران با هر سطحی از دانش فناوری آسان باشد. این شامل طراحی واکنش‌گرا و منوهای ساده آسان است.
				 
				 \item 
				 \textbf{روانشناسی رنگ‌ها:}
 می‌توان از دانش 
 \hyperref[ref:colorpsychology]{\textbf{روان‌شناسی رنگ‌ها}}
  برای ایجاد حس بهتر در کاربران هنگام استفاده از سیستم بهره برد.
				  
				  \item
				  \textbf{طراحی دوبعدی:}
				  اصول
				 \hyperref[ref:flatdesign]{\textbf{ طراحی دوبعدی}}
				  \footnote{\lr{\hyperref[ref:flatdesign]{\textbf{Flat-Design}}}}
				  بایستی موردتوجه قرار گیرد.
				    
				  \item   
				  \textbf{دسترس‌پذیری برای افراد دارای محدودیت‌های بینایی:}
				   سیستم باید به‌گونه‌ای طراحی شود که برای افراد دارای محدودیت‌های بینایی نیز قابل‌دسترس و استفاده باشد.
				   
				  \item 
				  \textbf{مقیاس‌پذیری:}
 سیستم باید قادر به پشتیبانی از تعداد زیادی کاربر و حجم بالای تراکنش‌ها را داشته باشد.این امر با بهره‌گیری از رایانش ابری و پایگاه‌داده‌های مقیاس‌پذیر حاصل می‌شود.
				   
				  \item 
				  \textbf{انعطاف‌پذیری:}
سیستم باید قابلیت افزودن ویژگی‌های جدید و ادغام با سیستم‌های دیگر را داشته باشد. این امر می‌تواند از طریق معماری \lr{\hyperref[ref:microservice]{\textbf{Microservice}}}‌ها و \lr{\hyperref[ref:api]{\textbf{API}}}های باز تسهیل شود.

				 \item 
				 \textbf{پشتیبانی:}
 ارائه خدمات پشتیبانی مؤثر و به‌موقع برای کمک به کاربران در استفاده از سیستم و حل مشکلات احتمالی ضروری است.
			 \end{itemize}

			\subsubsection{صفات سیستم نرم‌افزاری}

			ویژگی‌های کیفیت نرم‌افزار، معیارهایی هستند که برای 
			ارزیابی عملکرد یک محصول نرم‌افزاری توسط کارشناسان 
			آزمون نرم‌افزار استفاده می‌شوند؛ این ویژگی‌ها نشان‌دهنده 
			کیفیت و کارایی نرم‌افزار هستند.
			\\
			برخی از این ویژگی‌ها عبارت‌اند از:
			
			\begin{itemize}
				\item 
				\textbf{امنیت:}
				ازآنجاکه سامانه دادیار به اطلاعات شخصی کاربران دسترسی دارد امنیت از اهمیت بالایی برخوردار است. برخی از مواردی که باید در امنیت سامانه در نظر گرفت شوند عبارت‌اند از 
				\hyperref[ref:hash]{\textbf{درهم سازی}}
				 کردن رمز عبور کاربران، امکان فعال‌کردن ورود دومرحله‌ای، استفاده از 
				 \hyperref[ref:https]{\textbf{HTTPS}}
				  که به‌وسیله پروتکل 
				  \hyperref[ref:tls]{\textbf{TLS}}
				   رمزگذاری شده است.
				
				\item 
				\textbf{در دسترس بودن:}
				سیستم باید دسترسی بیست و چهار ساعته داشته باشد و بتواند به بهترین شکل ممکن کار خود را انجام دهد مگر در موارد اضطراری به‌روزرسانی و تعمیر که زمان در دسترس نبودن از قبل اطلاع‌رسانی شده است. سیستم باید به‌گونه‌ای طراحی شود که درحدامکان هنگام به‌روزرسانی و تعمیرات از دسترس خارج نشود.
				
				\item 
				\textbf{قابل اطمینان بودن:}
				 سیستم باید عملکردی مناسب داشته باشد و تاحدامکان خالی از هرگونه خطاهای نرم‌افزاری\footnote{\lr{Bug}} باشد. سیستم باید به‌صورت خودکار در وقفه زمانی‌های مشخص شده از پایگاه‌داده پشتیبان بگیرد تا درصورتی‌که اطلاعات آسیب دیدند از طریق نسخه‌های پشتیبان بازیابی شوند. سیستم باید در تمامی شرایط از جمله ورودی‌های نامعتبر، تعداد درخواست‌های زیاد و …. عملکرد مناسب خود را حفظ کند. به‌روزرسانی‌های سیستم نباید در یک‌زمان یکسان برای همه کاربران اعمال شوند تا در صورت ایجاد مشکل در نسخه جدید، سیستم به‌سرعت قابلیت تعمیر باشد و قابلیت اطمینان بودن سیستم افزایش یابد.
				
				\item 
				\textbf{قابلیت استفاده:}
				قابلیت استفاده به‌سهولت یادگیری و استفاده از یک نرم‌افزار اشاره دارد. نرم‌افزاری که از قابلیت استفاده بالایی برخوردار باشد؛ به کاربر اجازه می‌دهد تا به‌سرعت و بدون نیاز به آموزش و راهنمایی، وظایف خود را انجام دهد. رابط کاربری بصری، ساختار منظم و منطقی، و وجود راهنمایی و توضیحات کافی، از جمله عواملی هستند که به افزایش قابلیت استفاده نرم‌افزار کمک می‌کنند.
				
				از مواردی که باید در طراحی سامانه دادیار رعایت شوند عبارت‌اند از:
				\begin{enumerate}
					\item 
					\textbf{سهولت یادگیری:}
					سامانه دادیار باید به‌راحتی قابل‌یادگیری و استفاده باشد.
					
					\item
					\textbf{رابط کاربری:}
					\hyperref[ref:gui]{\textbf{رابط کاربری}}
					 دادیار باید بصری و 
					\hyperref[ref:userfriendly]{\textbf{کاربرپسند}}
					 باشد.
					
					\item
					\textbf{سازگاری:}
					 سامانه دادیار باید با 
					 \hyperref[ref:os]{\textbf{سیستم‌عامل‌ها}}
و \hyperref[ref:browser]{\textbf{مرورگرهای}} 
مختلف سازگار باشد. همچنین این سامانه باید به‌صورت 
					  \hyperref[ref:responsivedesign]{\textbf{واکنش‌گرا}}\footnote{\hyperref[ref:responsivedesign]{\textbf{\lr{Responsive Design}}}}
 طراحی شود تا با دستگاه‌های متفاوت با اندازه صفحه‌های رایج سازگار باشد.
				\end{enumerate}
				
				\item 
				\textbf{قابلیت نگهداری:}
				دادیار باید به‌گونه‌ای توسعه داده شود تا در صورت نیاز در آینده به‌راحتی توسعه‌پذیر باشد.
				
				از جمله مواردی که باید در طراحی دادیار رعایت شوند عبارت‌اند از:
				\begin{enumerate}
					\item 
					\textbf{خوانایی کد:}
					کد نرم‌افزار باید خوانا و قابل‌فهم باشد.
					
					\item 
					\textbf{قابلیت آزمودن\footnote{آزمودن، مصوب فرهنگستان \lr{Test} است.}:}
					نرم‌افزار باید به‌راحتی قابل آزمودن و اشکال‌زدایی باشد.
					
					\item 
					\textbf{قابلیت ارتقا:}
					نرم‌افزار باید به‌راحتی قابل ارتقا و اضافه‌کردن ویژگی‌های جدید باشد.
				\end{enumerate}
			\end{itemize}

	\newpage

	\section{قوانین کسب‌وکار}
\vspace{-2em} 
\par\noindent\rule{\textwidth}{0.72pt}

این دستورالعمل‌ها، توضیحاتی هستند که چگونگی اجرای فرایندهای معین و محدودیت‌های احتمالی که باید توسط سیستم رعایت شوند را مشخص می‌کنند. در ادامه به بیان مفصل این قوانین می‌پردازیم:

\begin{enumerate}
    \item \textbf{دسته‌بندی کاربران}: بر اساس نیاز سیستم و ماهیت پرونده‌ها، سطوح کاربری به چهار دسته زیر تقسیم شده است:
    \begin{itemize}
        \item \textbf{مدیر سامانه}: مسئول نظارت کلی بر سیستم و اعمال تغییرات مدیریتی.
        \item \textbf{شاکی}: فرد یا نهادی که ادعای خود را در سامانه ثبت می‌کند.
        \item \textbf{مشتکی عنه}: فرد یا نهادی که علیه او شکایت شده است.
        \item \textbf{مسئول پرونده}: کارشناسانی که وظیفه بررسی و داوری ادله دیجیتال را بر عهده دارند.
    \end{itemize}

    \item \textbf{نحوه ثبت‌نام و احراز هویت کاربران}:
    \begin{itemize}
        \item ثبت‌نام کاربران از طریق پست الکترونیک یا شماره همراه معتبر و رمز عبور شخصی صورت می‌گیرد.
        \item شاکیان و متشاکیان جهت احراز هویت باید مدارک شناسایی خود را ارائه دهند.
        \item مسئولان پرونده باید دارای مجوز رسمی از مراجع قانونی باشند.
        \item مدیر سامانه می‌تواند دسترسی کاربران را بررسی و در صورت لزوم محدود کند.
    \end{itemize}

    \item \textbf{بارگذاری و مدیریت ادله دیجیتال}:
    \begin{itemize}
        \item کاربران می‌توانند مدارک دیجیتال خود را در قالب‌های استاندارد تعیین‌شده در سیستم بارگذاری کنند.
        \item هر مدرک بارگذاری‌شده باید دارای مشخصات متادیتای لازم باشد (تاریخ، منبع، نوع فایل و ...).
        \item مسئول پرونده موظف است اصالت و صحت مدارک را بررسی کرده و در صورت نیاز، مدارک تکمیلی درخواست کند.
        \item هر مدرک تا پایان رسیدگی به پرونده در سیستم ذخیره می‌شود و پس از آن مطابق قوانین حذف یا بایگانی می‌گردد.
    \end{itemize}

    \item \textbf{امنیت و حفظ محرمانگی اطلاعات}:
    \begin{itemize}
        \item تمامی ارتباطات و اطلاعات کاربران از طریق پروتکل‌های رمزگذاری‌شده منتقل می‌شود.
        \item دسترسی به اطلاعات پرونده فقط برای کاربران مجاز امکان‌پذیر است.
        \item کاربران موظف‌اند از اطلاعات حساب کاربری خود محافظت کرده و از به اشتراک‌گذاری آن خودداری کنند.
        \item در صورت تخلف از قوانین حریم خصوصی، حساب کاربری متخلف تعلیق خواهد شد.
    \end{itemize}

    \item \textbf{نحوه رسیدگی به پرونده‌ها}:
    \begin{itemize}
        \item پس از ثبت شکایت، پرونده توسط مسئول مربوطه بررسی و اولویت‌بندی می‌شود.
        \item متشاکی موظف است در مدت تعیین‌شده به شکایت پاسخ دهد و ادله لازم را ارائه کند.
        \item نتایج بررسی و تصمیم نهایی از طریق سامانه به طرفین اطلاع‌رسانی خواهد شد.
    \end{itemize}

    \item \textbf{محدودیت‌های زمانی و دسترسی}:
    \begin{itemize}
        \item کاربران می‌توانند حداکثر تا ۷ روز پس از ثبت شکایت، مدارک تکمیلی بارگذاری کنند.
        \item دسترسی به پرونده‌های مختومه تنها تا ۶ ماه پس از پایان رسیدگی امکان‌پذیر است.
        \item در صورت درخواست مراجع قضایی، اطلاعات مورد نیاز ارائه خواهد شد.
    \end{itemize}

    \item \textbf{شرایط حذف یا تعلیق حساب کاربری}:
    \begin{itemize}
        \item در صورت ارسال اطلاعات نادرست یا جعلی، حساب کاربری فرد متخلف مسدود خواهد شد تحت پیگرد قانونی قرار خواهد گرفت.
        \item تخلفات متعدد از قوانین سامانه، منجر به تعلیق یا حذف دائمی حساب کاربری خواهد شد.
        \item کاربران در صورت اعتراض به تصمیمات سامانه، می‌توانند درخواست بازبینی ارائه دهند.
    \end{itemize}

    \item \textbf{سایر قوانین و مقررات}:
    \begin{itemize}
        \item هرگونه استفاده نادرست از سیستم، پیگرد قانونی خواهد داشت.
        \item قوانین و مقررات سامانه ممکن است بر اساس تغییرات قانونی به‌روزرسانی شود و کاربران موظف به رعایت نسخه جدید قوانین خواهند بود.
    \end{itemize}
\end{enumerate}

	\newpage
	\section{برنامه تکرار و برنامه مرحله}
\vspace{-2em} 
\par\noindent\rule{\textwidth}{0.72pt}

در این پروژه ما از 
متدولوژی یکنواخت چابک\footnote{\lr{Agile Unified Methodology}}
استفاده می‌کنیم. در ادامه نیازمندی‌های کارکردی را اولویت‌بندی و سپس پیش‌نیازهای هر نیازمندی را مشخص می‌کنیم. در نهایت در جدول \ref{table:longtable1} تعداد مراحل تکرار و نیازمندی‌های مختص به هر تکرار را مشخص می‌کنیم. 

\begin{longtable}[c]{|c|c|c|c|c|c|}
    \caption{برنامه تکرار و برنامه مرحله}
    \label{table:longtable1} \\
    \hline
    نیازمندی & اولویت(1/2/3) & وابستگی‌ها & تکرار اول & تکرار دوم & تکرار سوم \\
    \hline
    \endfirsthead
    
    \multicolumn{6}{c}%
    {{\bfseries \tablename\ \thetable{} -- ادامه از صفحه قبلی}} \\
    \hline
    نیازمندی & اولویت(1/2/3) & وابستگی‌ها & تکرار اول & تکرار دوم & تکرار سوم \\
    \hline
    \endhead
    
    \hline \multicolumn{6}{|r|}{{ادامه در صفحه بعدی}} \\ \hline
    \endfoot
    
    \hline
    \endlastfoot
    
    R1 & 1 & - & $\checkmark$ & & \\
    \hline
    R2 & 1 & - & $\checkmark$ & & \\
    \hline
    R3 & 2 & R1 - R2 & & $\checkmark$ & \\
    \hline
    R4 & 2 & R3 & & $\checkmark$ & \\
    \hline
    R5 & 2 & R3 & & $\checkmark$ & \\
    \hline
    R6 & 2 & R3 & & $\checkmark$ & \\
    \hline
    R7 & 3 & R5 & & & $\checkmark$ \\
    \hline
    R8 & 1 & - & $\checkmark$ & & \\
    \hline
    R9 & 1 & R8 & $\checkmark$ & & \\
    \hline
    R10 & 2 & R9 & & $\checkmark$ & \\
    \hline
    R11 & 2 & R9 & & $\checkmark$ & \\
    \hline
    R12 & 3 & R10 - R11 & & & $\checkmark$ \\
    \hline
    R13 & 2 & R9 & & $\checkmark$ & \\
    \hline
    R14 & 2 & R9 & & $\checkmark$ & \\
    \hline
    R15 & 2 & R9 & & $\checkmark$ & \\
    \hline
    R16 & 2 & R9 & & $\checkmark$ & \\
    \hline
    R17 & 3 & R16 & & & $\checkmark$ \\
    \hline
    R18 & 3 & R16 & & & $\checkmark$ \\
    \hline
    R19 & 3 & R16 & & & $\checkmark$ \\
    \hline
    R20 & 3 & R16 & & & $\checkmark$ \\
    \hline
    R21 & 3 & R16 & & & $\checkmark$ \\
    \hline
    R22 & 1 & - & $\checkmark$ & & \\
    \hline
    R23 & 1 & R22 & $\checkmark$ & & \\
    \hline
    R24 & 2 & R22 & & $\checkmark$ & \\
    \hline
    R25 & 1 & - & $\checkmark$ & & \\
    \hline
    R26 & 2 & R25 & & $\checkmark$ & \\
    \hline
    R27 & 2 & R25 & & $\checkmark$ & \\
    \hline
    R28 & 3 & R26 - R27 & & & $\checkmark$ \\
    \hline
    R29 & 2 & R25 & & $\checkmark$ & \\
    \hline
    R30 & 2 & R25 & & $\checkmark$ & \\
    \hline
    R31 & 3 & R30 & & & $\checkmark$ \\
    \hline
    R32 & 3 & R30 & & & $\checkmark$ \\
    \hline
    R33 & 3 & R30 & & & $\checkmark$ \\
    \hline
    R34 & 2 & R25 & & $\checkmark$ & \\
    \hline
    R35 & 2 & R25 & & $\checkmark$ & \\
    \hline
    R36 & 3 & R35 & & & $\checkmark$ \\
    \hline
    R37 & 3 & R35 & & & $\checkmark$ \\
    \hline
    R38 & 1 & - & $\checkmark$ & & \\
    \hline
    R39 & 2 & R38 & & $\checkmark$ & \\
    \hline
    R40 & 3 & R39 & & & $\checkmark$ \\
    \hline
    R41 & 2 & - & & $\checkmark$ & \\
    \hline
    R42 & 3 & R41 & & & $\checkmark$ \\
    \hline
    R43 & 2 & - & & $\checkmark$ & \\
    \hline
    R44 & 3 & R43 & & & $\checkmark$ \\
    \hline
    R45 & 2 & - & & $\checkmark$ & \\
    \hline
    R46 & 3 & R45 & & & $\checkmark$ \\
    \hline
    R47 & 2 & - & & $\checkmark$ & \\
    \hline
    R48 & 3 & R47 & & & $\checkmark$ \\
    \hline
    R49 & 2 & - & & $\checkmark$ & \\
    \hline
    R50 & 1 & R49 & $\checkmark$ & & \\
    \hline
    R51 & 2 & - & & $\checkmark$ & \\
    \hline
    R52 & 3 & R51 & & & $\checkmark$ \\
    \hline
    R53 & 2 & - & & $\checkmark$ & \\
\end{longtable}

\includepdf[pages=-]{resources/RACI.pdf}

\end{document}